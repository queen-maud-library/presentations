\documentclass{beamer}
\definecolor{links}{HTML}{2A1B81}

\usepackage[utf8]{inputenc}
\usepackage{hyperref}
\hypersetup{
  colorlinks=true,
  linkcolor=,
  urlcolor=links
}

\usetheme{Copenhagen}
\usecolortheme{seahorse}

\title{Bibliotekorientering}
\author{Thomas Rambø\inst{1}}
\institute[DMMH-biblioteket]{
  \inst{1}
  Biblioteket\\
  Dronning Mauds Minne Høgskole
}

\date[DMMH 2018]{DMMH, høsten 2018}
\logo{\includegraphics[height=1.5cm]{media/logo.png}}

\begin{document}

\frame{\titlepage}
\begin{frame}
  \frametitle{Innhold}
  \tableofcontents
\end{frame}

\section{Hvilke tjenester kan biblioteket tilby?}
\begin{frame}
  \frametitle{Pensumlitteratur og annen støttelitteratur}
  Bibliotekets samling består blant annet av \alert{pensumlitteratur} og annen \alert{støttelitteratur}. Det er flere eksemplarer tilgjengelig, og som regel ett eksemplar som bare kan leses i bibliotekets lokaler.

  \begin{block}{Bemerkning}
    Bruk søketjenesten Oria til å finne fram litteraturen.
  \end{block}
\end{frame}
  
\begin{frame}
  \frametitle{Vitenskapelig arkiv}
  I den digitale samlingen har vi et vitenskapelig arkiv med forskningsresultater fra DMMH og fremragende studentoppgaver.
\end{frame}

\end{document}
